% !TeX spellcheck = fr_FR
\chapter{Théorème des résidus et applications au calul d'intégrales réelles}


\section{Théorème des résidus}

\subsection{Énoncé}

\begin{theorem}
    Soient $D \subset \Cx$ un ouvert simplement connexe, $\gamma$ une courbe simple fermée régulière (par morceaux) contenue dans $D$ orientée positivement et $z_1, z_2, \ldots, z_m \in \inte \gamma$ tels que $z_i \neq z_j$ pour $i \neq j$.
    Si une fonction $f: D\setminus \{z_1, z_2, \ldots, z_m\} \rightarrow \Cx$ est holomorphe, alors:
    
    \[ \int_\gamma f(z) \dz = 2\pi i \sum_{k = 1}^m \Res_{z_k}(f) \]
\end{theorem}

\begin{note}
    Si une fonction est holomorphe sauf peut-être en un nombre fini de points $z_1, z_2,$ $ \ldots, z_m$, alors l'intégrale de $f$ le long de n'importe quelle courbe simple fermée régulière $\gamma$ contenue dans $D$ et orientée positivement est donnée par la somme (multipliée par $2\pi i$) des résidus de la fonction aux points $z_k$ (où $f$ n'est pas holomorphe) enfermés à l'intérieur de $\gamma$.
\end{note}

% 5.2.1??
\subsection{Exemples}

\begin{example}[1]
    Soit
    
    \[ f(z) = \frac{2}{z} + \frac{3}{z - 1} + \frac{1}{z^2} \]
    
    et $\gamma \subset \Cx$ une courbe régulière simple fermée orientée positivement.
    
    Discuter $\int_\gamma f(z) \dz$ en fonction de $\gamma$.
    
    \[ f(z) = \frac{p(z)}{q(z)} = \frac{2z(z-1) + 3z^2 + z -1 }{z^2 (z - 1)} \]
    
    $z_1 = 0$ est un pôle d'ordre 2 de $f$ ($p(0) \neq 0$, $q(0) = 0$, $q'(0) = 0$ et $q''(0) \neq 0$).
    
    $z_2 = 1$ est un pôle d'ordre 1 de $f$ ($p(1) \neq 0$, $q(1) = 0$ et $q'(1) \neq 0$).
    
    On a 
    
    \begin{align*}
    \Res_{z_1}(f) &= \Res_{0}(f) = \lim_{z \rightarrow 0} \frac{\dd}{\dz} \left[ z^2 f(z) \right] = \lim_{z \rightarrow 0} \frac{\dd}{\dz} \left[ 2z + \frac{3z^2}{z - 1} + 1 \right] \\&= \lim_{z \rightarrow 0} \left[ 2 + \frac{6z(z-1) - 3z^2}{(z - 1)^2} \right] = 2 + 0 = 2
    \end{align*}
    
    \begin{align*}
    \Res_{z_2}(f) &= \Res_{1}(f) = \lim_{z \rightarrow 1} \left[ (z - 1) f(z) \right] = \lim_{z \rightarrow 1} \left[ \frac{2(z-1)}{z} + \frac{3(z-1)}{z - 1} + \frac{z -1}{z^2} \right] \\&= 0 + 3 + 0 = 3
    \end{align*}
    
    \textbf{Distinction de cinq cas}
    
    \begin{description}
    \item[1\ier{} cas:] 0 et $1 \in \inte \gamma$
    
    \[ \int_\gamma f(z) \dz = 2\pi i \left[ \Res_{0}(f) + \Res_{1}(f) \right] = 2\pi i \left[ 2 + 3 \right] = 10 \pi i \]
    
    \item[2\ieme{} cas:] $0 \in \inte \gamma$ mais $1 \notin \adh{\gamma}$
    
    \[ \int_\gamma f(z) \dz = 2\pi i \ \Res_{0}(f) = 2\pi i \cdot 2 = 4 \pi i \]
    
    \item[3\ieme{} cas:] $0 \notin \adh{\gamma}$ mais $1 \in \inte \gamma$
    
    \[ \int_\gamma f(z) \dz = 2\pi i \ \Res_{1}(f) = 2\pi i \cdot 3 = 6 \pi i \]
    
    \item[4\ieme{} cas:] 0 et $1 \notin \adh{\gamma}$
    
    \[ \int_\gamma f(z) \dz = 0 \quad \textrm{(Théorème de Cauchy)} \]
    
    \item[5\ieme{} cas:] 0 ou $1 \in \gamma$
    
    L'intégrale $\int_\gamma f(z) \dz$ n'est pas définie.
    \end{description}

    \textit{Exemples: ex 2 et 3, série 9}
\end{example}

\begin{example}[2]
    Soit
    
    \[ f(z) = e^{\frac{1}{z^2}} \]
    
    et $\gamma$ une courbe simple fermée régulière orientée positivement.
    
    Discuter la valeur de $\int_\gamma f(z) \dz$ en fonction de $\gamma$.
    
    $f$ n'est pas holomorphe en $z_1 = 0$.
    Au voisinage de $z_1 = 0$, on a:
    
    \[ Lf(z) = \sum_{n = 0}^\infty \frac{1}{n!} \left(\frac{1}{z^2}\right)^n =
    1 + \sum_{n = 1}^\infty \frac{1}{n!z^{2n}} \]
    
    $\implies z_1 = 0$ est une singularité essentielle \textit{(cf. §4.2.2)}.
    
    On a que $\Res_{0}(f) := c_{-1} = 0$ (pas de terme $z^{-1}$ dans $Lf(z)$).
    
    \textbf{Distinction de trois cas}
    
    \begin{description}
        \item[1\ier{} cas:] $0 \in \inte \gamma$
        
        \[ \int_\gamma f(z) \dz = 2\pi i \ \Res_{0}(f) = 2\pi i \ 0 = 0 \]
        
        (mais $f$ n'est pas holomorphe en $z = 0$)
        
        \item[2\ieme{} cas:] $0 \notin \adh{\gamma}$
        
        \[ \int_\gamma f(z) \dz = 0  \quad \textrm{(Théorème de Cauchy)} \]
        
        $f$ holomorphe dans un domaine $D$ simplement connexe tel que $\adh{\gamma} \subset D \implies \gamma \subset \adh{\gamma} \subset D$
        
        \item[3\ieme{} cas:] $0 \in \gamma$
        
        L'intégrale $\int_\gamma f(z) \dz$ n'est pas définie.
    \end{description}
\end{example}

\subsection{Démonstration du Théorème des résidus}

\begin{proof}
    Soient $\gamma_k$ avec $k = 1, 2, \ldots, m$ $m$ courbes simples fermées régulières orientées positivement contenues dans $\inte \gamma$ et contenant $z_k$ dans leur intérieur.
    
    Comme $f: \adh{\gamma} \setminus \bigcup\limits_{k = 1}^m \inte \gamma_k \rightarrow \Cx$ est holomorphe, en appliquant le corollaire du Théorème de Cauchy \textit{(cf. §3.2.4)}, on a:
    
    \[ \int_\gamma f(z) \dz = \sum_{k = 1}^m \int_{\gamma_k} f(z) \dz := 2\pi i \sum_{k = 1}^m \Res_{z_k}(f) \]
    
    L'intégrale dans la somme est le coefficient $c_{-1}$ de la série de Laurent de $f$ au voisinage de $z_k$ multipliée par $2\pi i$.
    La deuxième égalité est donnée par la définition du résidu de $f$ en $z_k$ \textit{(§4.2.2)}.
\end{proof}

\begin{remark}
    Si $f$ est holomorphe dans $D$, alors pour toute courbe simple $\gamma$ fermée régulière dans $D$, il n'y a aucune singularité $z_k \in \inte \gamma$.
    Dans ce cas, $\sum\limits_{k = 1}^m \Res_{z_k}(f) = 0$ et le Théorème des résidus donne le résultat $\int_\gamma f(z) \dz = 0$ du Théorème de Cauchy \textit{(§3.2.1)}.
\end{remark}


\section{Application du Théorème des résidus au calcul d'intégrales réelles}

\subsection{Calcul d'intégrale de fonctions périodiques}

\begin{enumerate}[label=\alph*)]
    \item But: calculer des intégrales de la forme
    
    \[ \int_0^{2\pi} f(\cos \theta, \sin \theta) \dth \]
    
    avec $f: \R^2 \rightarrow \R$ définie par
    
    \[ (x,y) \longmapsto f(x,y) = \frac{p(x,y)}{q(x,y)} \]
    
    où $p$ et $q$ sont des fonctions polynômiales avec $q(\cos \theta, \sin \theta) \neq 0 \ \forall \theta \in [0, 2\pi]$
    
    \item Méthode:
    
    \begin{itemize}
    \item 
    On pose $z = e^{i\theta}$ et on a donc
    
    \[ \cos \theta = \frac{e^{i\theta} + e^{-i\theta}}{2}
    = \frac{1}{2} \left(z + \frac{1}{z}\right) \]
    
    \[ \sin \theta = \frac{e^{i\theta} - e^{-i\theta}}{2i}
    = \frac{1}{2i} \left(z - \frac{1}{z}\right) \]
    
    \item 
    On définit:
    
    \[ \tilde{f}: \Cx \longrightarrow \Cx \ ; \ z \longmapsto \tilde{f}(z) = \frac{1}{iz} f\left(\frac{1}{2} \left(z + \frac{1}{z}\right), \frac{1}{2i} \left(z - \frac{1}{z}\right)\right) \]
    
    On considère $\gamma$ le cercle \textit{unité} centré en $z = 0$ orienté positivement et $z_k$ pour $k = 1, 2, \ldots, m$ les singularités de $\tilde{f}$ à \textit{l'intérieur} de $\gamma$.
    
    $z_k \notin \gamma$ car $q(\cos \theta, \sin \theta) \neq 0$ pour $\theta \in [0, 2\pi] \implies$ pas de singularité de $\tilde{f}$ \textit{sur} $\gamma$
    
    \item 
    On applique le Théorème des résidus à la fonction $ \tilde{f}$ intégrée le long de $\gamma$:
    
    \[ \int_\gamma \tilde{f}(z) \dz = 2\pi i \sum_{k = 1}^m \Res_{z_k} (\tilde{f}) \]
    
    Mais on remarque que:
    
    \begin{align*}
    \int_\gamma \tilde{f}(z) \dz &= \int_\gamma \frac{1}{iz} f\left(\frac{1}{2} \left(z + \frac{1}{z}\right), \frac{1}{2i} \left(z - \frac{1}{z}\right)\right) \dz
    \\&= \int_0^{2\pi} \frac{1}{ie^{i\theta}} f(\cos \theta, \sin \theta) i e^{i\theta} \dth
    \\&= \int_0^{2\pi} f(\cos \theta, \sin \theta) \dth
    \end{align*}
    
    est exactement l'intégrale \textit{réelle} que l'on veut calculer.
    \end{itemize}

    Le résultat est:
    
    \[ \int_0^{2\pi} f(\cos \theta, \sin \theta) \dth = 2\pi i \sum_{k = 1}^m \Res_{z_k}(\tilde{f}) \]
    
    où $z_k$ pour $k = 1, 2, \ldots, m$ sont les singularités de $\tilde{f}$ à l'intérieur du cercle unité $\gamma$ centré en $z = 0$.
    
    \item Exemples
    
    \begin{example}[1]
        Calculer
        
        \[ \int_0^{2\pi} \frac{\dth}{\sqrt{5} - \sin \theta} \]
        
        On a $f(\cos \theta, \sin \theta) = \frac{1}{\sqrt{5} - \sin \theta}$, et $\sqrt{5} - \sin \theta \neq 0$ pour $\theta \in [0, 2\pi]$ et
        
        \[ \tilde{f} := \frac{1}{iz} f\left(\frac{1}{2} \left(z + \frac{1}{z}\right), \frac{1}{2i} \left(z - \frac{1}{z}\right)\right) = \frac{1}{iz} \frac{1}{\sqrt{5} - \frac{1}{2i} \left(z - \frac{1}{z}\right)} = \frac{2}{-z^2 + 2i\sqrt{5}z + 1} \]
        
        Les singularités de $\tilde{f}$ sont les zéros de $-z^2 + 2i\sqrt{5}z + 1$. $\Delta = (2i\sqrt{5})^2 + 4 = -16$
        
        \[ z_1 = \frac{-2i\sqrt{5} + 4i}{-2} = i(\sqrt{5} - 2) \]
        
        \[ z_2 = \frac{-2i\sqrt{5} - 4i}{-2} = i(\sqrt{5} + 2) \]
        
        On a que $-z^2 + 2i\sqrt{5}z + 1 = -(z - z_1)(z - z_2) = - \left[ z - i(\sqrt{5} - 2) \right] \left[ z - i(\sqrt{5} + 2) \right]$ et
        
        \[ \tilde{f}(z) = \frac{-2}{\left[ z - i(\sqrt{5} - 2) \right] \left[ z - i(\sqrt{5} + 2) \right]} \]
        
        Soit $\gamma$ le cercle unité centré en $z = 0$ et orienté positivement.
        
        \[ 0 < \Imp z_1 = \sqrt{5} - 2 < 1 \implies z_1 \in \inte \gamma  \]
        
        \[ \Imp z_2 = \sqrt{5} + 2 > 1 \implies z_2 \notin \inte \gamma \]
        
        On a:
        
        \[ \int_0^{2\pi} \frac{\dth}{\sqrt{5} - \sin \theta} = 2\pi i \ \Res_{z_1} (\tilde{f}) \]
        
        $z_1 = i(\sqrt{5} - 2)$ est un pôle d'ordre 1 de $\tilde{f}$
        
        \begin{align*}
        \Res_{z_1} (\tilde{f})& = \lim_{z \rightarrow z_1} (z - z_1) \tilde{f}(z) \\&= \lim_{z \rightarrow i(\sqrt{5} - 2)} (z - i(\sqrt{5} - 2)) \frac{-2}{\left[ z - i(\sqrt{5} - 2) \right] \left[ z - i(\sqrt{5} + 2) \right]}
        \\&= \frac{-2}{-4i} = \frac{1}{2i}
        \end{align*}
        
        Le résultat est
        
        \[ \int_0^{2\pi} \frac{\dth}{\sqrt{5} - \sin \theta} = 2 \pi i \frac{1}{2i} = \pi \]
    \end{example}

    \begin{example}[2]
        Calculer:
        
        \[ \int_0^{2\pi} \frac{\dth}{2 + \cos \theta} \]
        
        On a $f(\sin \theta, \cos \theta) = \frac{1}{2 + \cos \theta} \quad 2 + \cos \theta \neq 0 \ \forall \theta \in [0, 2\pi]$
        
        \[ \tilde{f} := \frac{1}{iz} f\left(\frac{1}{2} \left(z + \frac{1}{z}\right), \frac{1}{2i} \left(z - \frac{1}{z}\right)\right) = \frac{1}{iz} \frac{1}{2 + \frac{1}{2} \left(z + \frac{1}{z}\right)} = \frac{2}{i \left(z^2 + 4z + 1\right)} \]
        
        Les singularités de $\tilde{f}$ sont les zéros de $z^2 + 4z + 1$. $\Delta = 12$
        
        \[ z_1 = \frac{-4 + 2\sqrt{3}}{2} = \sqrt{3} - 2 \]
        \[ z_2 = \frac{-4 - 2\sqrt{3}}{2} = -(\sqrt{3} + 2) \]
        
        On a $z^2 +  4z + 1 = (z - z_1)(z - z_2) = (z + 2 - \sqrt{3})(z + 2 + \sqrt{3})$ et
        
        \[ \tilde{f} = \frac{2}{i(z + 2 - \sqrt{3})(z + 2 + \sqrt{3})} \]
        
        Soit $\gamma$ le cercle unité centré en $z = 0$ orienté positivement.
        
        \[ -1 < z_1 = \sqrt{3} - 2 < 0 \implies  z_1 \in \inte \gamma \]
        
        \[ z_2 = -(\sqrt{3} + 2) < -1 \implies  z_2 \notin \inte \gamma \]
        
        Donc
        
        \[ \int_0^{2\pi} \frac{\dth}{2 + \cos \theta} = 2\pi i \Res_{z_1} (\tilde{f}) \]
        
        $z_1 = \sqrt{3} - 2$ est un pôle d'ordre 1 de $\tilde{f}$, donc:
        
        \begin{align*}
        \Res_{z_1} (\tilde{f}) &= \lim_{z \rightarrow z_1} (z - z_1) \tilde{f}(z)
        \\&= \lim_{z \rightarrow \sqrt{3} - 2} (z + 2 - \sqrt{3}) \frac{2}{i(z + 2 - \sqrt{3})(z + 2 + \sqrt{3})}
        \\&= \frac{1}{i\sqrt{3}}
        \end{align*}
        
        Le résultat est:
        
        \[ \int_0^{2\pi} \frac{\dth}{2 + \cos \theta} = 2\pi i \frac{1}{i\sqrt{3}} = \frac{2\pi}{\sqrt{3}} \]
        
        \textit{Autres exemples: ex 1 à 4, série 10 et ex 1, série 11}
    \end{example}
\end{enumerate}

\subsection{Calcul d'intégrales généralisées}

\begin{enumerate}
\item But: calculer des intégrales de la forme

\[ \int_{-\infty}^\infty f(x) e^{i\alpha x} \dx \quad \alpha \in \R_+ \ (\alpha \geq 0),\ f: \R \rightarrow \R \]

Où $f(x) = \frac{p(x)}{q(x)}$, et $p, q$ sont des fonctions polynômiales telles que $q(x) \neq 0 \ \forall x \in \R$ et degré$(q) - $ degré$(p) \geq 2$.

\begin{remark}
    Les conditions sur $p$ et $q$ impliquent que l'intégrale généralisée $\int_{-\infty}^\infty |f(x)| \dx$ existe.
\end{remark}

\item Méthode: on choisit un nombre réel $r > 0$ et on considère la courbe $\gamma_r := L_r \cup C_r$ orientée positivement, où:

\begin{itemize}
    \item $L_r$ est le segment de droite $[-r,r]$ situé sur \textit{l'axe réel},
    \item $C_r$ est le demi-cercle de rayon $r$ centré en $z = 0$ et situé dans le \textit{demi-plan supérieur}.
\end{itemize}

$\gamma_r = L_r \cup C_r$ est une courbe simple fermée par morceaux.

On définit la fonction $g: \Cx \rightarrow \Cx$ par $z \mapsto g(z) = f(z) e^{i\alpha z} = \frac{p(z)}{q(z)} e^{i\alpha z}$.

\begin{constatation}
    Les seules singularités de $g$ sont les zéros de $q$.
    Par hypothèse, $q$ est une fonction polynômiale et $q(x) \neq 0 \ \forall x \in R$, alors $q$ possède un nombre \textit{fini} de zéros et aucun ne se trouve situé sur l'axe réel.
\end{constatation}

\begin{idea}
    On choisit $r > 0$ suffisamment grand pour que \textbf{tous} les zéros de $q$ situés dans le \textit{demi-plan supérieur} soient à l'intérieur de $\gamma_r$.
\end{idea}

En appliquant le Théorème des résidus à $g(z) = f(z) e^{i\alpha z}$ intégrée le long de $\gamma_r$, on a:

\[ \int_{\gamma_r} f(z) e^{i\alpha z} \dz = 2\pi i \sum_{k = 1}^m \Res_{z_k}(g) \]

où $z_k$ pour $k = 1, 2, \ldots, m$ sont les singularités de $f$ (i.e. les zéros de $q$) situées dans le demi-plan supérieur.

D'autre part, puisque $\gamma_r = L_r \cup C_r$, on a:

\[ \int_{\gamma_r} f(z) e^{i\alpha z} \dz = \int_{L_r} f(z) e^{i\alpha z} \dz + \int_{C_r} f(z) e^{i\alpha z} \dz \]

\[ \lim_{r \rightarrow \infty} \int_{\gamma_r} f(z) e^{i\alpha z} \dz
= \lim_{r \rightarrow \infty} \int_{L_r} f(z) e^{i\alpha z} \dz
+ \lim_{r \rightarrow \infty} \int_{C_r} f(z) e^{i\alpha z} \dz \]

\textbf{Étude de chaque limite}

\begin{enumerate}
    \item
    
    \[
        \lim_{r \rightarrow \infty} \int_{\gamma_r} f(z) e^{i\alpha z} \dz
        = \lim_{r \rightarrow \infty} \left[ 2\pi i \sum_{k = 1}^m \Res_{z_k}(g) \right]
        = 2\pi i \sum_{k = 1}^m \Res_{z_k}(g)
    \]
    
    \item 
    
    \[
        \lim_{r \rightarrow \infty} \int_{L_r} f(z) e^{i\alpha z} \dz
        = \lim_{r \rightarrow \infty} \int_{-r}^r f(x) e^{i\alpha x} \dx
        = \int_{-\infty}^\infty f(x) e^{i\alpha x} \dx
    \]
    
    où $z = x \in [-r, r] \subset \R$.
    Ceci est l'intégrale généralisée que l'on veut calculer!
    
    \item On montre que si $\deg(q) - \deg(p) \geq 2$, alors:
    
    \[
        \lim_{r \rightarrow \infty} \int_{C_r} f(z) e^{i\alpha z} \dz = 0
    \]
\end{enumerate}

\begin{result}[final]
    On obtient la formule suivante:
    
    \[ \int_{-\infty}^\infty f(x) e^{i\alpha x} \dx
    = 2\pi i \sum_{k = 1}^m \Res_{z_k}(g) \]
    
    où $g(z) = f(z)e^{i\alpha z}$ et $z_k$ pour $k = 1, 2, \ldots, m$ sont les singularités de $f$ situées dans le demi-plan supérieur (i.e. les zéros de $q$ tels que $\Imp z_k > 0$).
\end{result}

\item

\begin{example}[1]
Calculer

\[ \int_{-\infty}^\infty \frac{x^2}{x^4 + 16} \dx \]

où $\alpha = 0$ et $f(x) = \frac{x^2}{x^4 + 16}$.
On a $p(x) = x^2$ et $q(x) = x^4 + 16$.
Conditions vérifiées: $q(x) \neq 0 \ \forall x \in \R$ et $\deg(q) - \deg(p) = 2$.

Calcul avec la méthode des résidus avec $g(z) = f(z) = \frac{z^2}{z^4 + 16}$.

Singularités de $f(z) \iff$ recherche des zéros de $q(z)$

$q(z) = 0 \iff z^4 + 16 = 0 \iff z^4 = -16 \iff 16 e^{i\pi} \iff z = 2 e^{i\left(\frac{\pi}{4} + \frac{2n\pi}{4}\right)}$ avec $n = 0,1,2,3$.

Les singularités sont:

\begin{description}
    \item[pour n = 0]
        \[ z_1 = 2 e^{i\frac{\pi}{4}}
        = 2 \left( \cos \frac{\pi}{4} + i \sin \frac{\pi}{4}\right)
        = 2 \left( \frac{\sqrt{2}}{2} + i \frac{\sqrt{2}}{2} \right) = \sqrt{2} (1 + i)\]

    \item[pour n = 1]
        \[ z_2 = 2 e^{i\frac{3\pi}{4}} = \ldots = \sqrt{2} (-1 + i) \]

    \item[pour n = 2]
        \[ z_2 = 2 e^{i\frac{5\pi}{4}} = \ldots = - \sqrt{2} (1 + i) \]

    \item[pour n = 3]
        \[ z_2 = 2 e^{i\frac{7\pi}{4}} = \ldots = - \sqrt{2} (-1 + i) \]
\end{description}

Ce sont des pôles d'ordre 1, car $z^4 + 16 = (z - z_1) (z - z_2) (z - z_3) (z - z_4)$.

Les seuls pôles qui contribuent sont $z_1$ et $z_2$ situés dans le demi-plan \textit{supérieur}.

Calcul des résidus de $f$ en $z_1$ et $z_2$:

\[
    \Res_{z_1}(f) = \frac{p(z_1)}{q'(z_1)}
    = \frac{z_1^2}{4 z_1^3} = \frac{1}{4 z_1} = \frac{1}{4\sqrt{2}(1+i)} = \frac{1 - i}{8\sqrt{2}}
\]

\[
    \Res_{z_2}(f) = \frac{p(z_2)}{q'(z_2)}
    = \frac{z_2^2}{4 z_2^3} = \frac{1}{4 z_2} = \frac{1}{4\sqrt{2}(-1+i)} = - \frac{1 + i}{8\sqrt{2}}
\]

L'intégrale vaut:

\[
    \int_{-\infty}^\infty \frac{x^2}{x^4 + 16} \dx
    = 2 \pi i \left[ \Res_{z_1}(f) + \Res_{z_2}(f) \right]
    = 2 \pi i \left[ \frac{1 - i}{8\sqrt{2}} - \frac{1 + i}{8\sqrt{2}} \right] = \frac{\sqrt{2}\pi}{4}
\]

\end{example}


\begin{example}[2]
    Calculer
    
    \[ \int_{-\infty}^\infty \frac{\cos 5x}{x^2 + 1} \dx \]
    
    On utilise la formule d'Euler
    
    \[ e^{i5x} = \cos 5x + i \sin 5x \]
    
    et on peut écrire
    
    \[
        \int_{-\infty}^\infty \frac{\cos 5x}{x^2 + 1} \dx
        = \Rep \left[ \int_{-\infty}^\infty \frac{e^{i5x}}{x^2 + 1} \dx \right]
    \]
    
    On considère $\int_{-\infty}^\infty f(x) e^{i\alpha x} \dx$ où $\alpha = 5$ et $f(x) = \frac{1}{x^2 + 1}$.
    
    On a $p(x) = 1$ et $q(x) = x^2 + 1$.
    Conditions vérifiées: $q(x) \neq 0 \ \forall x \in \R$ et $\deg(p) - \deg(q) = 2$.
    
    Méthode des résidus avec $g(z) = f(z) e^{i5z} = \frac{e^{i5z}}{z^2 + 1}$.
    
    Singularités de $f \implies$ zéros de $q \implies q(z) = 0 \implies z^2 + 1 = 0 \implies z^2 = -1 \implies z_1 = i$ et $z_2 = -i$.
    Ce sont des pôles d'ordre 1 et $z^2 + 1 = (z - i)(z + i)$.
    
    Le seul pôle qui contribue est $z_1 = i$ situé dans le demi-plan supérieur.
    
    \[
        \Res_{z_1}(g)
        = \lim_{z \rightarrow z_1} (z - z_1) g(z)
        = \lim_{z \rightarrow i} (z - i) \frac{e^{i5z}}{(z - i)(z + i)}
        = \frac{e^{i5i}}{2i}
        = \frac{e^{-5}}{2i}
    \]
    
    Alors:
    
    \[
        \int_{-\infty}^\infty \frac{e^{i5x}}{x^2 + 1} \dx = 2\pi i \Res_i (g) = 2\pi i \frac{e^{-5}}{2i} = \frac{\pi}{e^{5}}
    \]
    
    et
    
    \[
        \int_{-\infty}^\infty \frac{\cos 5x}{x^2 + 1} \dx
        = \Rep \left[ \int_{-\infty}^\infty \frac{e^{i5x}}{x^2 + 1} \dx \right]
        = \Rep \left[ \frac{\pi}{e^{5}} \right]
        = \frac{\pi}{e^{5}}
    \]
    
    \textit{Autres exemples: ex. 2 et 3, série 11}
\end{example}

\end{enumerate}

\begin{proof}
    Preuve de
    
    \[ \lim_{r \rightarrow \infty} \int_{C_r} f(z) e^{i\alpha z} \dz = 0 \]
    
    Sur $C_r$, on a $z = r e^{i\theta}$ avec $\theta \in [0,\pi]$ et $\dz = i r e^{i\theta} \dth$
    
    \begin{align*}
        \implies \int_{C_r} f(z) e^{i\alpha z} \dz
        &= \int_0^\pi f(r e^{i\theta}) e^{i\alpha r e^{i\theta}} i r e^{i\theta} \dth
        \\&= ir \int_0^\pi f(r e^{i\theta}) e^{i\alpha r (\cos \theta + i \sin \theta)} e^{i\theta} \dth
        \\&= ir \int_0^\pi f(r e^{i\theta}) e^{-\alpha r\sin \theta}e^{i\alpha r \cos \theta} e^{i\theta} \dth
        \\&= ir \int_0^\pi f(r e^{i\theta}) e^{-\alpha r\sin \theta}e^{i \left(\alpha r \cos \theta + \theta\right)} \dth
    \end{align*}
    
    Alors
    
    \[
        \left| \int_{C_r} f(z) e^{i\alpha z} \dz \right|
        \leq r \int_0^\pi \left| f(r e^{i\theta}) \right| e^{-\alpha r\sin \theta} \left| e^{i \left(\alpha r \cos \theta + \theta\right)} \right| \dth
    \]
    
    \begin{enumerate}
        \item 
        Puisque $f(z) = \frac{p(z)}{q(z)}$ avec $\deg(q) - \deg(p) \geq 2$, alors on a que $\left|f(z)\right| \leq \frac{C}{|z|^2}$ pour tout $z \in \Cx$ avec $|z|$ suffisamment grand, $C \in \R^\ast_+$ est une constante.
        
        \item 
        Comme $\alpha \geq 0$, $r > 0$ et $\sin \theta \geq 0$ pour $\theta \in [0, \pi]$, on a que $-\alpha r \sin \theta \leq 0$.
        
        \[ 0 \leq e^{-\alpha r \sin \theta} \leq 1 \]
        
        \item 
        De plus, $\left| e^{i(\theta + \alpha r \cos \theta)} \right| = 1$ car $\left|e^{ix}\right| = 1$ pour tout $x \in \R$.
        Donc on obtient:
        
        \[
            \left| \int_{C_r} f(z) e^{i\alpha z} \dz \right|
            \leq r \int_0^\pi \frac{C}{r^2} \dth = \frac{C}{r} \int_0^\pi \dth = \frac{C\pi}{r}
        \]
        
        Puisque 
        
        \[
            \lim_{r \rightarrow \infty} \frac{C\pi}{r} = 0 \implies \lim_{r \rightarrow \infty} \int_{C_r} f(z) e^{i\alpha z} \dz = 0
        \]
    \end{enumerate}
\end{proof}

\begin{remark}[1]
    Pour $z = r e^{i\theta} = r(cos \theta + i \sin \theta)$, on a $\module{e^{i\alpha z}} = e^{-\alpha r \sin \theta}$.
    Lorsque $\alpha \leq 0$, on a $-\alpha r \sin \theta \leq 0$ et $\module{e^{i\alpha z}} < 1$ si $\theta \in [\pi, 2\pi]$ et il faut donc choisir le demi-cercle $C_r$ dans le demi-plan pour avoir
    
    \[ \lim_{r \rightarrow \infty} \int_{C_r} f(z) e^{i\alpha z} \dz = 0 \]
    
    Résultat: pour calculer des intégrales généralisées $\int_{-\infty}^\infty f(x) e^{i\alpha x} \dx$ avec $\alpha \in \R_-$, on applique la même méthode en considérant les singularités de $f$ situées dans le \textit{demi-plan inférieur}.
    
    \textbf{Attention} à l'orientation positive de $\gamma_r = L_r \cup C_r$.
    On obtient l'intégrale suivante:
    
    \[ \int_{\infty}^{-\infty} f(x) e^{i\alpha x} \dx \]
\end{remark}

\begin{remark}[2]
    La méthode permet de calculer la transformée de Fourier d'une fonction $f$ du type quotient de polynômes vérifiant les conditions demandées.
    
    Pour $\alpha \geq 0$
    
    \[ \hat{f}(-\alpha) := \ostpi \int_{-\infty}^\infty f(x)e^{i\alpha x} \dx = 2\pi i \sum_{k = 1}^m \Res_{z_k}(g) \]
    
    où $g(z) = f(z) e^{i\alpha z}$ et $z_k = 1, \ldots, m$ sont les singularités de $f$ situées dans le demi-plan supérieur.
    
    Pour $\alpha \leq 0$, on considère les singularités de $f$ dans le demi-plan inférieur et on obtient $-\hat{f}(-\alpha)$.
\end{remark}
