% !TeX spellcheck = fr_FR
\chapter{Fonctions holomorphes et équations de Cauchy-Riemann}


\section{Introduction}

\subsection{Motivation}

\begin{description}
    \item[But:] étendre l'étude de fonctions réelles (du type $f: \R \rightarrow \R$) à des fonctions qui dépendent d'\textbf{une} variable complexe qui sont à valeurs complexes (du type $f: \C \rightarrow \C$ où $\C$ est l'ensemble des nombres complexes).
    
    \item[Rôle:] établir les notions de limite, de continuité, de dérivabilité et d'intégration dans $\C$.
    
    \item[Intérêt:] méthodes puissantes qui permettent de calculer facilement des intégrales \textbf{ré\-elles} compliquées.
\end{description}

\textit{Cf. ex. 4, série 3}

\subsection{Rappel sur les nombres complexes}

\begin{itemize}
    \item 
    $\C$ désigne l'ensemble des nombres complexes
    \item 
    $z \in \C \iff z = x + iy$ avec $x = \Rep z \in \R$ et $y = \Imp z \in \R$ et $i^2 = -1$
    \item 
    $\C^* = \C \setminus \{0\}$ où $0 = 0 + i0$
    \item 
    complexe conjugué de $z \quad \conj{z} = x - iy$ 
    \item 
    module de $z \in \C \quad \module{z} = \sqrt{x^2 + y^2} \in \R_+$
    \item 
    représentation polaire de $z \in \C^* \quad z = \module{z} e^{i\theta} = \module{z} (\cos \theta + i \sin \theta)$
    \item 
    $\theta$ est appelé l'argument de $z$ et est noté $\arg z$
\end{itemize}

\begin{remark}
    Pour $z \in \C^*$:
    \begin{itemize}
        \item 
        L'argument de $z$ est défini à $2k\pi$ près avec $k \in \Z$
        \item 
        Par convention, la \textbf{valeur (détermination) principale} de l'argument de $z$ est l'unique angle $\theta \in ] -\pi ; \pi ]$ tel que $\frac{z}{\module{z}} = \cos \theta + i \sin \theta$
    \end{itemize}
\end{remark}


\section{Fonctions complexes}

\subsection{Définitions}

\begin{definition}
    Une \textbf{fonction d'une variable complexe} à valeur dans $\C$ s'écrit:
    
    \[
    f: \begin{array}{{rcl}}
    \C & \longrightarrow & \C \\
    z = x+iy & \longmapsto & f(z) = u(x,y) + i v(x,y) \\
    \end{array}
    \]
    
    où
    
    \[
    u: \begin{array}{{rcl}}
    \R^2 & \longrightarrow & \R \\
    (x,y) & \longmapsto & u(x,y) \\
    \end{array}
    \textrm{ et }\enspace
    v: \begin{array}{{rcl}}
    \R^2 & \longrightarrow & \R \\
    (x,y) & \longmapsto & v(x,y) \\
    \end{array}
    \]
    
    sont deux fonctions à valeurs réelles qui s'appellent respectivement la partie réelle de $f$ (on note $u = \Rep f$) et la partie imaginaire de $f$ (on note $v = \Imp f$).
\end{definition}

\begin{remark}
    Les variables $x \in \R$ et $y \in \R$ des fonctions $u$ et $v$ sont les parties réelles et imaginaires de la variable $z \in \C$ de la fonction $f$.
\end{remark}

\subsection{Exemples}

\begin{example}\hfill
\begin{enumerate}[label=\arabic{enumi})]
    \item 
    \[
    f: \begin{array}{{rcl}}
    \C & \longrightarrow & \C \\
    z = x+iy & \longmapsto & f(z) = \conj{z} = x - iy \\
    \end{array}
    \]
    
    On a $u(x,y) = x \enspace$ et $v(x,y)=-y$.
    
    \item 
    \[
    f: \begin{array}{{rcl}}
    \C & \longrightarrow & \C \\
    z = x+iy & \longmapsto & f(z) = z^2 = (x + iy)^2 = x^2 - y^2 + 2ixy \\
    \end{array}
    \]
    
    On a $u(x,y) = x^2 - y^2 \enspace$ et $v(x,y)= 2xy$.
    
    \item 
    \[
    f: \begin{array}{{rcl}}
    \C^* & \longrightarrow & \C \\
    z = x+iy & \longmapsto & f(z) = \frac{1}{z}
    = \frac{1}{x + iy}
    = \frac{x - iy}{(x + iy)(x - iy)}
    = \frac{x - iy}{x^2 + y^2} \\
    \end{array}
    \]
    
    On a $u(x,y) = \frac{x}{x^2 + y^2} \enspace$ et $v(x,y)= -\frac{y}{x^2 + y^2}$.
    
    \item 
    Pour $z = x + iy \in \C$, la fonction exponentielle est définie par:
    
    \[e^z = e^{x + iy} := e^x(\cos y + i \sin y) \in \C^*\]
    
    On a $u(x,y) = e^x \cos y \enspace$ et $v(x,y)= e^x \sin y$.
    
    \begin{remark}
    Contrairement au cas réel, $e^z$ n'est pas bijective sur $\C$ car $e^{z + 2ik\pi} = e^z$ $\forall z \in \Z$ \textit{(ex.4, série 3)}.
    En choisissant $y$ tel que $-\pi < y \leq \pi$, la fonction $e^z$ est bijective sur l'ensemble $\left\{z \in \C : \Imp z \in \enspace ] -\pi; \pi]\right\}$.
    Avec cette convention, c'est la \textbf{\og restriction bijective de l'exponentielle complexe\fg{}}.
    \end{remark}

    \item 
    Pour $z \in \C^*$, la fonction logarithme est définie par:
    
    \[
    \log z := \ln \module{z} + i \arg z
    \]
    
    avec le choix de la valeur principale $\arg z \in \enspace ]-\pi;\pi[$. Avec cette convention, c'est la \textbf{\og détermination principale du logarithme complexe \fg} (correspond à la fonction réciproque de la restriction bijective de l'exponentielle).\\
    En écrivant $z = x + iy$, on a $u(x,y) = \ln \module{x + iy} = \ln \sqrt{x^2 + y^2} \enspace$ et $v(x,y)= \arg (x + iy)$.
    
    \begin{remark}\hfill
    \begin{enumerate}[label=\alph*)]
    \item 
    Les formules valables en analyse réelle ne sont pas nécessairement valables en analyse complexe.
    Par exemple, en général, on a $\log(z_1 z_2) \neq \log z_1 + \log z_2$.
    En effet, pour $z_1 = -1$ et $z_2 = -1$:
    
    \[\log\left[(-1)(-1)\right] = \log(1) := \ln \module{1} + i \arg(1) = 0 + i0 = 0\]
    
    mais
    
    \[\log(-1) + \log(-1) = 2 \log(-1) := 2 \left[\ln \module{-1} + i \arg (-1) \right] = 2i\pi \neq 0 \]
    
    \item 
    La fonction $\log z$ n'est pas continue sur le demi-axe réel négatif.
    En effet, par exemple pour $z = -1$, on considère $t > 0$:
    
    \[z^+_t = -1 + it \enspace \textrm{ et }\enspace
     z^-_t = -1 - it \]
     
    On a $\lim\limits_{t \rightarrow 0^+} z^+_t = -1$ et $\lim\limits_{t \rightarrow 0^+} z^-_t = -1$.

    \begin{note}
    Pour $z = x + iy$, on a la valeur principale:
    
    \[
    \arg z =
    \left\{
    \begin{array}{rcl}
    \pi + \Arctg\frac{y}{x} & \textrm{si} & x < 0, y > 0\\
    -\pi + \Arctg\frac{y}{x} & \textrm{si} & x < 0, y < 0\\
    \end{array}
    \right.
    \]
    \end{note}

    Avec la définition du logarithme, on a:
    
    \begin{align*}
    \lim\limits_{t \rightarrow 0^+} \log(z^+_t)
    &=
    \lim\limits_{t \rightarrow 0^+} \ln \module{-1 + it} + i \lim\limits_{t \rightarrow 0^+} \arg(-1 + it)
    \\&=
    \lim\limits_{t \rightarrow 0^+} \ln \sqrt{1 + t^2} + i \lim\limits_{t \rightarrow 0^+} \left[\pi + \Arctg(-t)\right]
    \\&=
    \ln (1) + i \pi = 0 + i\pi = i\pi
    \end{align*}
    
    \begin{align*}
    \lim\limits_{t \rightarrow 0^+} \log(z^-_t)
    &=
    \lim\limits_{t \rightarrow 0^+} \ln \module{-1 - it} + i \lim\limits_{t \rightarrow 0^+} \arg(-1 - it)
    \\&=
    \lim\limits_{t \rightarrow 0^+} \ln \sqrt{1 + t^2} + i \lim\limits_{t \rightarrow 0^+} \left[-\pi + \Arctg(t)\right]
    \\&=
    \ln (1) - i \pi = 0 - i\pi = -i\pi
    \end{align*}
    
    \textbf{Conclusion:} $\lim\limits_{t \rightarrow 0^+} \log(z^+_t) \neq \lim\limits_{t \rightarrow 0^+} \log(z^-_t) \implies$ la fonction $\log z$ n'est pas continue en $z = -1$.
    De façon analogue, on obtient le même résultat $\forall z \in \enspace ]-\infty;0[$.
    $\log z$ \textbf{n'est pas continue} pour $z \in \enspace ]-\infty;0]$
    
    \textbf{Résultat final:} En excluant le demi-axe réel négatif, on a que la fonction $\log z$ est continue sur l'ensemble:
    
    \[
    V = \C \enspace \setminus \enspace ]-\infty;0] = \C \setminus \left\{z \in \C : \Imp z = 0, \Rep z \leq 0\right\}
    \]
    
    C'est la \textbf{\og restriction continue du logarithme complexe \fg}.
    \end{enumerate}
    \end{remark}

    \item 
    Pour $z \in \C$, on définit les fonctions trigonométriques et hyperboliques:
    
    \begin{align*}
    \cos z = \frac{e^{iz} + e^{-iz}}{2}
    &\hspace{2cm} \sin z = \frac{e^{iz} - e^{-iz}}{2i}\\
    \cosh z = \frac{e^{z} + e^{-z}}{2}
    &\hspace{2cm} \sinh z = \frac{e^{z} - e^{-z}}{2}
    \end{align*}
    
    \textit{Cf. ex. 1, série 4}
\end{enumerate}
\end{example}


\section{Limites, continuité, dérivabilité}

\subsection{Définitions}

\begin{definition}
    Les notions de topologie (ouverts, fermés, etc.), de limite, de continuité et de dérivabilité sont analogues à celles de l'analyse réelle.
    Soit $f: \C \rightarrow \C$, alors:
    
    \begin{enumerate}[label=\arabic{enumi})]
    \item 
    $f$ possède une limite $l \in \C$ en $z_0 \in \C$ (notation $\lim\limits_{z \rightarrow z_0} f(z) = l$) si:\\
    $\forall\, \epsilon > 0 \enspace \exists\, \delta > 0 :\enspace 0 < \module{z - z_0} < \delta \implies \module{f(z) - l} < \epsilon$
    
    \item 
    $f$ est continue en $z_0 \in \C$ si $\lim\limits_{z \rightarrow z_0} f(z) = f(z_0)$
    
    \item 
    $f$ est dérivable en $z_0 \in \C$ si $\lim\limits_{z \rightarrow z_0} \frac{f(z) - f(z_0)}{z - z_0}$ existe et est finie.
    La limite s'appelle la dérivée de $f$ en $z_0$ et est notée $f'(z_0)$.
    Les règles de dérivation établies dans $\R$ sont valables dans $\C$.
    
    \item 
    Étant donné un ouvert $V \subset \C$, on dit que la fonction $f: V \rightarrow \C$ est \textbf{holomorphe} (ou analytique complexe) dans $V$ si $f$ est \textit{définie et dérivable} $\forall z \in V$.
    \end{enumerate}
\end{definition}

\subsection{Équations de Cauchy-Riemann}

\begin{remark}[sur un abus de notation]
    Étant donné un ouvert $V \subset \C$, on l'identifie souvent au sous-ensemble correspondant de $\R^2$, i.e. on écrit indifféremment $z = x + iy \in V (\in \C)$ ou $(x,y) \in V (\in \R^2)$ de façon abusive.
\end{remark}

%!TeX newpage for cosmetic reasons
\newpage

\begin{theorem}[de Cauchy-Riemann]
    Soit $V \in \C$ un ouvert et soit une fonction $f: V \rightarrow \C$, où $u: V \rightarrow \R$ et $v: V \rightarrow \R$ sont respectivement les parties réelles et imaginaires de $f$.
    Alors les deux affirmations suivantes sont équivalentes:
    \begin{enumerate}[label=\arabic{enumi})]
        \item 
        $f$ est holomorphe dans $V$
        \item 
        Les fonctions $u,v \in C^1(V)$ et satisfont les équations de Cauchy-Riemann:
        
        \[
        \frac{\partial u}{\partial x}(x,y) = \frac{\partial v}{\partial y}(x,y)
        \quad,\quad
        \frac{\partial u}{\partial y}(x,y) = - \frac{\partial v}{\partial x}(x,y)
        \]
        
        En particulier, si $f$ est holomorphe dans $V$, alors on a:
        
        \[
        f'(z) = \frac{\partial u}{\partial x}(x,y) + i \frac{\partial v}{\partial x}(x,y) = \frac{\partial v}{\partial y}(x,y) - i \frac{\partial u}{\partial y}(x,y) \quad \forall z = x + iy \in V
        \]
    \end{enumerate}
\end{theorem}

\begin{remark}\hfill
\begin{enumerate}[label=\arabic{enumi})]
    \item 
    Démonstration du Théorème: voir §2.3.4 (fin du chapitre)
    \item 
    Les équations de Cauchy-Riemann sont une condition nécessaire pour que $f$ soit holomorphe mais elles ne sont pas une condition suffisante.
    Si $u$ et $v$ sont continûment dérivables ($u,v \in C^1(V)$), alors elles deviennent une condition suffisante.
    \item 
    Utilité du Théorème: pour qu'une fonction $f$ soit holomorphe dans un ouvert $V$, il suffit de vérifier que les équations de Cauchy-Riemann pour $u = \Rep f \in C^1(V)$ et $v = \Imp f \in C^1(V)$ sont satisfaites dans $V$.
    Si les équations de Cauchy-Riemann ne sont pas vérifiées en $(x_0,y_0) \in V$, alors $f(z_0)$ n'est pas holomorphe en $z_0 = x_0 + iy_0$.
    \item 
    Pour alléger la notation, on écrit:
    
    \[
    u_x = \frac{\partial u}{\partial x}, \quad
    u_y = \frac{\partial u}{\partial y}, \quad
    v_x = \frac{\partial v}{\partial x}, \quad
    v_y = \frac{\partial v}{\partial y}
    \]
    
    Équations de Cauchy-Riemann:
    
    \[
    u_x = v_y \quad, \quad u_y = -v_x
    \]
    
    \textit{Exemples: ex. 1 à 4, série 4}
\end{enumerate}
\end{remark}

\subsection{Exemples}

\begin{example}[1]\hfill
    
    $f(z) = z^2$ définie pour $z = x + iy \in \C$.
    
    $f(z) = (x + iy) ^2 = x^2 - y^2 + i2xy$.
    
    \[\implies u(x,y) = x^2 - y^2, \quad v(x,y) = 2xy\]
    
    \[\left.
    \begin{array}{ll}
    u_x(x,y) = 2x &
    u_y(x,y) = -2y \\
    v_x(x,y) = 2y &
    v_y(x,y) = 2x
    \end{array}
    \right\} \implies
    \begin{array}{l}
    u_x = v_y \\
    u_y = -v_x
    \end{array}
    \forall (x,y) \in \C\]
    
    CR (Cauchy-Riemann) satisfaites $\forall z \in \C \implies f$ holomorphe dans $\C$.
    
    De plus: $f'(z) = u_x(x,y) + i v_x(x,y) = 2x + 2iy = 2(x + iy) = 2z$.
\end{example}

\begin{example}[2]\hfill
    
    $f(z) = \conj{z}$ définie pour $z = x + iy \in \C$.
    
    $f(z) = \conj{x + iy} = x - iy$.
    
    \[\implies u(x,y) = x, \quad v(x,y) = -y\]
    
    \[\left.
    \begin{array}{ll}
    u_x(x,y) = 1 &
    u_y(x,y) = 0 \\
    v_x(x,y) = 0 &
    v_y(x,y) = -1
    \end{array}
    \right\} \implies
    \begin{array}{l}
    u_x \neq v_y \\
    u_y = -v_x
    \end{array}\]
    
    CR non satisfaites $\implies f$ n'est pas holomorphe dans $\C$.
\end{example}

\begin{example}[3]\hfill
    
    $f(z) = e^z$ définie pour $z = x + iy \in \C$.
    
    $f(z) = e^x (\cos y + i \sin y) = e^x \cos y + i e^x \sin y$
    
    \[\implies u(x,y) = e^x \cos y, \quad v(x,y) = e^x \sin y\]
    
    \[\left.
    \begin{array}{ll}
    u_x(x,y) = e^x \cos y &
    u_y(x,y) = -e^x \sin y \\
    v_x(x,y) = e^x \sin y &
    v_y(x,y) = e^x \cos y
    \end{array}
    \right\} \implies
    \begin{array}{l}
    u_x = v_y \\
    u_y = -v_x
    \end{array}\]
    
    CR satisfaites $\implies f$ holomorphe dans $\C$.
    
    \[f'(z) = u_x(x,y) + i v_x(x,y) = e^x \cos y + i e^x \sin y := e^z\]
\end{example}

\begin{example}[4]\hfill
    
    $f(z) = \log z = \ln \module{z} + i \arg z$ avec la détermination principale définie pour $V = \C\, \setminus\, ]-\infty; 0] = \C \setminus \{z \in \C : \Imp z = 0, \Rep z \leq 0\}$.
    $\log z$ est holomorphe dans $V$ et on a:
    
    \[f'(z) = \frac{1}{z} \enspace \forall z \in V\]
    
    En effet: preuve pour le demi-plan $D = \{z \in \C : \Rep z > 0\}$.
    Pour $z = x +iy$ avec $x > 0$ et $y \in \R$.
    On a: $\module{z} = \sqrt{x^2 + y^2}$ et $\arg z = \Arctg \frac{y}{x}$.
    
    Donc $\log z := \ln \sqrt{x^2 + y^2} + i \Arctg \frac{y}{x}.$
    
    \[\implies u(x,y) = \ln \sqrt{x^2 + y^2}, \quad v(x,y) = \Arctg \frac{y}{x}\]
    
    \[\left.
    \begin{array}{ll}
    u_x(x,y) = \frac{\frac{1}{2}\frac{2x}{\sqrt{x^2 + y^2}}}{\sqrt{x^2 + y^2}}
    = \frac{x}{x^2 + y^2} &
    u_y(x,y) = \frac{\frac{1}{2}\frac{2y}{\sqrt{x^2 + y^2}}}{\sqrt{x^2 + y^2}}
    = \frac{y}{x^2 + y^2} \\
    v_x(x,y) = \frac{-\frac{y}{x^2}}{1 + \frac{y^2}{x^2}}
    = - \frac{y}{x^2 + y^2} &
    v_y(x,y) = \frac{\frac{1}{x}}{1 + \frac{y^2}{x^2}}
    =  \frac{x}{x^2 + y^2}
    \end{array}
    \right\} \implies
    \begin{array}{l}
    u_x = v_y \\
    u_y = -v_x
    \end{array}\]
    
    CR satisfaites $\implies f$ holomorphe dans $D$.
    
    De plus:
    
    \begin{align*}
    f'(z) &= u_x(x,y) + i v_x(x,y) = \frac{x}{x^2 + y^2} - i \frac{y}{x^2 + y^2}
    \\&=
    \frac{x - iy}{x^2 + y^2} = \frac{x - iy}{(x + iy)(x - iy)} = \frac{1}{x + iy} = \frac{1}{z}
    \end{align*}
\end{example}

\textit{Cf. ex. 1, série 5}

\subsection{Démonstration des équations de Cauchy-Riemann}

\begin{proof}{'$\implies$'}\hfill

    Soient $z_0 = x_0 + i y_0 \in V$ et $z = (x_0 + \alpha) + i (y_0 + \beta) \in V$ avec $\alpha, \beta \in \R$.
    Puisque $f$ est holomorphe dans $V$, alors $f'(z_0) := \lim\limits_{z \rightarrow z_0} \frac{f(z) - f(z_0)}{z - z_0}$ existe $\forall z_0 \in V$
    On a:
    
    \[\frac{f(z) - f(z_0)}{z - z_0} = \frac{[u(x_0 + \alpha, y_0 + \beta) + iv(x_0 + \alpha, y_0 + \beta)] - [u(x_0,y_0) + iv(x_0,y_0)]}{\alpha + i\beta}\]
    
    \begin{enumerate}[label=\alph*)]
    \item
    En posant $\beta = 0$, on obtient:
    
    \begin{align*}
    f'(z_0) &= \lim\limits_{\alpha \rightarrow 0} \frac{[u(x_0 + \alpha, y_0) + iv(x_0 + \alpha, y_0)] - [u(x_0,y_0) + iv(x_0,y_0)]}{\alpha}
    \\&=
    \lim\limits_{\alpha \rightarrow 0} \frac{u(x_0 + \alpha, y_0)  - u(x_0,y_0)}{\alpha} + i \lim\limits_{\alpha \rightarrow 0} \frac{v(x_0 + \alpha, y_0) - v(x_0,y_0)}{\alpha}
    \\&=
    u_x(x_0,y_0) + i v_x(x_0,y_0)
    \end{align*}
    
    \item
    En posant $\alpha = 0$, on obtient:
    
    \begin{align*}
    f'(z_0) &= \lim\limits_{\beta \rightarrow 0} \frac{u(x_0, y_0 + \beta) - u(x_0,y_0)}{i\beta} + i \lim\limits_{\beta \rightarrow 0} \frac{v(x_0, y_0 + \beta) - v(x_0,y_0)}{i\beta}
    \\&=
    \frac{1}{i}\, u_y(x_0,y_0) + v_y(x_0,y_0) = v_y(x_0,y_0) - i u_y(x_0,y_0)
    \end{align*}
    \end{enumerate}

    Les deux limites existent et sont identiques.

    \[\implies u_x(x_0,y_0) = v_y(x_0,y_0), \quad u_y(x_0,y_0) = -v_x(x_0,y_0) \quad \textrm{(équations de CR)}\]
    
    '$\Longleftarrow$' \hspace{1em} On utilise les développements de Taylor au 1\ier{} ordre de $u(x,y)$ et $v(x,y)$ pour montrer que $f'(z_0)$ existe.
    
    \textit{Cf. ex. 5, série 4 et ex. 2-4, série 5}
\end{proof}

\begin{remark}[finale]
    Affirmer que $f: \C \rightarrow \C$ est dérivable en $z = x + iy$ \textbf{n'est pas équivalent} au fait que le champ vectoriel $\tilde{f}: \R^2 \rightarrow \R^2$ est continûment dérivable dans le contexte usuel de $\R^2$ (i.e matrice jacobienne $\begin{pmatrix}u_x & u_y\\v_x & v_y\end{pmatrix}$ de $\tilde{f}$ existe avec $u_x, u_y, v_x, v_y$ continues).
    
    Par exemple, $f(z) = \conj{z}$ n'est pas holomorphe dans $\C$: si $z_0 = x_0 + i y_0$ et $z = (x_0 + \alpha) + i (y_0 + \beta)$, alors on a:
    
    \[\frac{f(z) - f(z_0)}{z - z_0} = \frac{(x_0 + \alpha) - i(y_0 + \beta) - x_0 + iy_0}{(x_0 + \alpha) + i(y_0 + \beta) - x_0 - iy_0} = \frac{\alpha - i\beta}{\alpha + i\beta} = \left\{\begin{array}{rl}-1 & \textrm{si } \alpha = 0\\1 & \textrm{si } \beta = 0\end{array}\right.\]
    
    \[\lim\limits_{\alpha \rightarrow 0} \frac{f(z) - f(z_0)}{z - z_0} = -1, \quad \lim\limits_{\beta \rightarrow 0} \frac{f(z) - f(z_0)}{z - z_0} = 1\]
    
    $\implies f'(z_0)$ n'existe pas.
    
    Mais $\tilde{f}:\begin{array}{rcl} \R^2 & \longrightarrow & \R^2\\ (x,y) & \longmapsto & (u(x,y),v(x,y)) = (x,-y)\end{array}$
    
    $\implies$ Matrice jacobienne $\begin{pmatrix}u_x & u_y\\v_x & v_y\end{pmatrix} = \begin{pmatrix}1 & 0\\0 & -1 \end{pmatrix}$ existe $\forall (x,y) \in \R^2$
\end{remark}
