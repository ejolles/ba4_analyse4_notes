% !TeX spellcheck = fr_FR
\chapter{Théorème des résidus et applications au calul d'intégrales réelles}


\section{Théorème des résidus}

\subsection{Énoncé}

\begin{theorem}
    Soient $D \subset \Cx$ un ouvert simplement connexe, $\gamma$ une courbe simple fermée régulière (par morceaux) contenue dans $D$ orientée positivement et $z_1, z_2, \ldots, z_m \in \inte \gamma$ tels que $z_i \neq z_j$ pour $i \neq j$.
    Si une fonction $f: D\setminus \{z_1, z_2, \ldots, z_m\} \rightarrow \Cx$ est holomorphe, alors:
    
    \[ \int_\gamma f(z) \dz = 2\pi i \sum_{k = 1}^m \Res_{z_k}(f) \]
\end{theorem}

\begin{note}
    Si une fonction est holomorphe sauf peut-être en un nombre fini de points $z_1, z_2,$ $ \ldots, z_m$, alors l'intégrale de $f$ le long de n'importe quelle courbe simple fermée régulière $\gamma$ contenue dans $D$ et orientée positivement est donnée par la somme (multipliée par $2\pi i$) des résidus de la fonction aux points $z_k$ (où $f$ n'est pas holomorphe) enfermés à l'intérieur de $\gamma$.
\end{note}

% 5.2.1??
\subsection{Exemples}

\begin{example}[1]
    Soit
    
    \[ f(z) = \frac{2}{z} + \frac{3}{z - 1} + \frac{1}{z^2} \]
    
    et $\gamma \subset \Cx$ une courbe régulière simple fermée orientée positivement.
    
    Discuter $\int_\gamma f(z) \dz$ en fonction de $\gamma$.
    
    \[ f(z) = \frac{p(z)}{q(z)} = \frac{2z(z-1) + 3z^2 + z -1 }{z^2 (z - 1)} \]
    
    $z_1 = 0$ est un pôle d'ordre 2 de $f$ ($p(0) \neq 0$, $q(0) = 0$, $q'(0) = 0$ et $q''(0) \neq 0$).
    
    $z_2 = 1$ est un pôle d'ordre 1 de $f$ ($p(1) \neq 0$, $q(1) = 0$ et $q'(1) \neq 0$).
    
    On a 
    
    \begin{align*}
    \Res_{z_1}(f) &= \Res_{0}(f) = \lim_{z \rightarrow 0} \frac{\dd}{\dz} \left[ z^2 f(z) \right] = \lim_{z \rightarrow 0} \frac{\dd}{\dz} \left[ 2z + \frac{3z^2}{z - 1} + 1 \right] \\&= \lim_{z \rightarrow 0} \left[ 2 + \frac{6z(z-1) - 3z^2}{(z - 1)^2} \right] = 2 + 0 = 2
    \end{align*}
    
    \begin{align*}
    \Res_{z_2}(f) &= \Res_{1}(f) = \lim_{z \rightarrow 1} \left[ (z - 1) f(z) \right] = \lim_{z \rightarrow 1} \left[ \frac{2(z-1)}{z} + \frac{3(z-1)}{z - 1} + \frac{z -1}{z^2} \right] \\&= 0 + 3 + 0 = 3
    \end{align*}
    
    \textbf{Distinction de cinq cas}
    
    \begin{description}
    \item[1\ier{} cas:] 0 et $1 \in \inte \gamma$
    
    \[ \int_\gamma f(z) \dz = 2\pi i \left[ \Res_{0}(f) + \Res_{1}(f) \right] = 2\pi i \left[ 2 + 3 \right] = 10 \pi i \]
    
    \item[2\ieme{} cas:] $0 \in \inte \gamma$ mais $1 \notin \adh{\gamma}$
    
    \[ \int_\gamma f(z) \dz = 2\pi i \ \Res_{0}(f) = 2\pi i \cdot 2 = 4 \pi i \]
    
    \item[3\ieme{} cas:] $0 \notin \adh{\gamma}$ mais $1 \in \inte \gamma$
    
    \[ \int_\gamma f(z) \dz = 2\pi i \ \Res_{1}(f) = 2\pi i \cdot 3 = 6 \pi i \]
    
    \item[4\ieme{} cas:] 0 et $1 \notin \adh{\gamma}$
    
    \[ \int_\gamma f(z) \dz = 0 \quad \textrm{(Théorème de Cauchy)} \]
    
    \item[5\ieme{} cas:] 0 ou $1 \in \gamma$
    
    L'intégrale $\int_\gamma f(z) \dz$ n'est pas définie.
    \end{description}

    \textit{Exemples: ex 2 et 3, série 9}
\end{example}

\begin{example}[2]
    Soit
    
    \[ f(z) = e^{\frac{1}{z^2}} \]
    
    et $\gamma$ une courbe simple fermée régulière orientée positivement.
    
    Discuter la valeur de $\int_\gamma f(z) \dz$ en fonction de $\gamma$.
    
    $f$ n'est pas holomorphe en $z_1 = 0$.
    Au voisinage de $z_1 = 0$, on a:
    
    \[ Lf(z) = \sum_{n = 0}^\infty \frac{1}{n!} \left(\frac{1}{z^2}\right)^n =
    1 + \sum_{n = 1}^\infty \frac{1}{n!z^{2n}} \]
    
    $\implies z_1 = 0$ est une singularité essentielle \textit{(cf. §4.2.2)}.
    
    On a que $\Res_{0}(f) := c_{-1} = 0$ (pas de terme $z^{-1}$ dans $Lf(z)$).
    
    \textbf{Distinction de trois cas}
    
    \begin{description}
        \item[1\ier{} cas:] $0 \in \inte \gamma$
        
        \[ \int_\gamma f(z) \dz = 2\pi i \ \Res_{0}(f) = 2\pi i \ 0 = 0 \]
        
        (mais $f$ n'est pas holomorphe en $z = 0$)
        
        \item[2\ieme{} cas:] $0 \notin \adh{\gamma}$
        
        \[ \int_\gamma f(z) \dz = 0  \quad \textrm{(Théorème de Cauchy)} \]
        
        $f$ holomorphe dans un domaine $D$ simplement connexe tel que $\adh{\gamma} \subset D \implies \gamma \subset \adh{\gamma} \subset D$
        
        \item[3\ieme{} cas:] $0 \in \gamma$
        
        L'intégrale $\int_\gamma f(z) \dz$ n'est pas définie.
    \end{description}
\end{example}

\subsection{Démonstration du Théorème des résidus}

\begin{proof}
    Soient $\gamma_k$ avec $k = 1, 2, \ldots, m$ $m$ courbes simples fermées régulières orientées positivement contenues dans $\inte \gamma$ et contenant $z_k$ dans leur intérieur.
    
    Comme $f: \adh{\gamma} \setminus \bigcup\limits_{k = 1}^m \inte \gamma_k \rightarrow \Cx$ est holomorphe, en appliquant le corollaire du Théorème de Cauchy \textit{(cf. §3.2.4)}, on a:
    
    \[ \int_\gamma f(z) \dz = \sum_{k = 1}^m \int_{\gamma_k} f(z) \dz := 2\pi i \sum_{k = 1}^m \Res_{z_k}(f) \]
    
    L'intégrale dans la somme est le coefficient $c_{-1}$ de la série de Laurent de $f$ au voisinage de $z_k$ multipliée par $2\pi i$.
    La deuxième égalité est donnée par la définition du résidu de $f$ en $z_k$ \textit{(§4.2.2)}.
\end{proof}

\begin{remark}
    Si $f$ est holomorphe dans $D$, alors pour toute courbe simple $\gamma$ fermée régulière dans $D$, il n'y a aucune singularité $z_k \in \inte \gamma$.
    Dans ce cas, $\sum\limits_{k = 1}^m \Res_{z_k}(f) = 0$ et le Théorème des résidus donne le résultat $\int_\gamma f(z) \dz = 0$ du Théorème de Cauchy \textit{(§3.2.1)}.
\end{remark}


\section{Application du Théorème des résidus au calcul d'intégrales réelles}

\subsection{Calcul d'intégrale de fonctions périodiques}

\begin{enumerate}[label=\alph*)]
    \item But: calculer des intégrales de la forme
    
    \[ \int_0^{2\Pi} f(\cos \theta, \sin \theta) \dth \]
    
    avec $f: \R^2 \rightarrow \R$ définie par
    
    \[ (x,y) \longmapsto f(x,y) = \frac{p(x,y)}{q(x,y)} \]
    
    où $p$ et $q$ sont des fonctions polynômiales avec $q(\cos \theta, \sin \theta) \neq 0 \ \forall \theta \in [0, 2\pi]$
    
    \item Méthode:
    
    \begin{itemize}
    \item 
    On pose $z = e^{i\theta}$ et on a donc
    
    \[ \cos \theta = \frac{e^{i\theta} + e^{-i\theta}}{2}
    = \frac{1}{2} \left(z + \frac{1}{z}\right) \]
    
    \[ \sin \theta = \frac{e^{i\theta} - e^{-i\theta}}{2i}
    = \frac{1}{2i} \left(z - \frac{1}{z}\right) \]
    
    \item 
    On définit:
    
    \[ \tilde{f}: \Cx \longrightarrow \Cx \ ; \ z \longmapsto \tilde{f}(z) = \frac{1}{iz} f\left(\frac{1}{2} \left(z + \frac{1}{z}\right), \frac{1}{2i} \left(z - \frac{1}{z}\right)\right) \]
    
    On considère $\gamma$ le cercle \textit{unité} centré en $z = 0$ orienté positivement et $z_k$ pour $k = 1, 2, \ldots, m$ les singularités de $\tilde{f}$ à \textit{l'intérieur} de $\gamma$.
    
    $z_k \notin \gamma$ car $q(\cos \theta, \sin \theta) \neq 0$ pour $\theta \in [0, 2\pi] \implies$ pas de singularité de $\tilde{f}$ \textit{sur} $\gamma$
    
    \item 
    On applique le Théorème des résidus à la fonction $ \tilde{f}$ intégrée le long de $\gamma$:
    
    \[ \int_\gamma \tilde{f}(z) \dz = 2\pi i \sum_{k = 1}^m \Res_{z_k} (\tilde{f}) \]
    
    Mais on remarque que:
    
    \begin{align*}
    \int_\gamma \tilde{f}(z) \dz &= \int_\gamma \frac{1}{iz} f\left(\frac{1}{2} \left(z + \frac{1}{z}\right), \frac{1}{2i} \left(z - \frac{1}{z}\right)\right) \dz
    \\&= \int_0^{2\pi} \frac{1}{ie^{i\theta}} f(\cos \theta, \sin \theta) i e^{i\theta} \dth
    \\&= \int_0^{2\pi} f(\cos \theta, \sin \theta) \dth
    \end{align*}
    
    est exactement l'intégrale \textit{réelle} que l'on veut calculer.
    \end{itemize}

    Le résultat est:
    
    \[ \int_0^{2\pi} f(\cos \theta, \sin \theta) \dth = 2\pi i \sum_{k = 1}^m \Res_{z_k}(\tilde{f}) \]
    
    où $z_k$ pour $k = 1, 2, \ldots, m$ sont les singularités de $\tilde{f}$ à l'intérieur du cercle unité $\gamma$ centré en $z = 0$.
    
    \item Exemples
    
    \begin{example}[1]
        Calculer
        
        \[ \int_0^{2\Pi} \frac{\dth}{\sqrt{5} - \sin \theta} \]
        
        On a $f(\cos \theta, \sin \theta) = \frac{1}{\sqrt{5} - \sin \theta}$, et $\sqrt{5} - \sin \theta \neq 0$ pour $\theta \in [0, 2\pi]$ et
        
        \[ \tilde{f} := \frac{1}{iz} f\left(\frac{1}{2} \left(z + \frac{1}{z}\right), \frac{1}{2i} \left(z - \frac{1}{z}\right)\right) = \frac{1}{iz} \frac{1}{\sqrt{5} - \frac{1}{2i} \left(z - \frac{1}{z}\right)} = \frac{2}{-z^2 + 2i\sqrt{5}z + 1} \]
        
        Les singularités de $\tilde{f}$ sont les zéros de $-z^2 + 2i\sqrt{5}z + 1$. $\Delta = (2i\sqrt{5})^2 + 4 = -16$
        
        \[ z_1 = \frac{-2i\sqrt{5} + 4i}{-2} = i(\sqrt{5} - 2) \]
        
        \[ z_2 = \frac{-2i\sqrt{5} - 4i}{-2} = i(\sqrt{5} + 2) \]
        
        On a que $-z^2 + 2i\sqrt{5}z + 1 = -(z - z_1)(z - z_2) = - \left[ z - i(\sqrt{5} - 2) \right] \left[ z - i(\sqrt{5} + 2) \right]$ et
        
        \[ \tilde{f}(z) = \frac{-2}{\left[ z - i(\sqrt{5} - 2) \right] \left[ z - i(\sqrt{5} + 2) \right]} \]
        
        Soit $\gamma$ le cercle unité centré en $z = 0$ et orienté positivement.
        
        \[ 0 < \Imp z_1 = \sqrt{5} - 2 < 1 \implies z_1 \in \inte \gamma  \]
        
        \[ \Imp z_2 = \sqrt{5} + 2 > 1 \implies z_2 \notin \inte \gamma \]
        
        On a:
        
        \[ \int_0^{2\pi} \frac{\dth}{\sqrt{5} - \sin \theta} = 2\pi i \ \Res_{z_1} (\tilde{f}) \]
        
        $z_1 = i(\sqrt{5} - 2)$ est un pôle d'ordre 1 de $\tilde{f}$
        
        \begin{align*}
        \Res_{z_1} (\tilde{f})& = \lim_{z \rightarrow z_1} (z - z_1) \tilde{f}(z) \\&= \lim_{z \rightarrow i(\sqrt{5} - 2)} (z - i(\sqrt{5} - 2)) \frac{-2}{\left[ z - i(\sqrt{5} - 2) \right] \left[ z - i(\sqrt{5} + 2) \right]}
        \\&= \frac{-2}{-4i} = \frac{1}{2i}
        \end{align*}
        
        Le résultat est
        
        \[ \int_0^{2\pi} \frac{\dth}{\sqrt{5} - \sin \theta} = 2 \pi i \frac{1}{2i} = \pi \]
    \end{example}

    \begin{example}[2]
        Calculer:
        
        \[ \int_0^{2\pi} \frac{\dth}{2 + \cos \theta} \]
        
        On a $f(\sin \theta, \cos \theta) = \frac{1}{2 + \cos \theta} \quad 2 + \cos \theta \neq 0 \ \forall \theta \in [0, 2\pi]$
        
        \[ \tilde{f} := \frac{1}{iz} f\left(\frac{1}{2} \left(z + \frac{1}{z}\right), \frac{1}{2i} \left(z - \frac{1}{z}\right)\right) = \frac{1}{iz} \frac{1}{2 + \frac{1}{2} \left(z + \frac{1}{z}\right)} = \frac{2}{i \left(z^2 + 4z + 1\right)} \]
        
        Les singularités de $\tilde{f}$ sont les zéros de $z^2 + 4z + 1$. $\Delta = 12$
        
        \[ z_1 = \frac{-4 + 2\sqrt{3}}{2} = \sqrt{3} - 2 \]
        \[ z_1 = \frac{-4 - 2\sqrt{3}}{2} = -(\sqrt{3} + 2) \]
        
        On a $z^2 +  4z + 1 = (z - z_1)(z - z_2) = (z + 2 - \sqrt{3})(z + 2 + \sqrt{3})$ et
        
        \[ \tilde{f} = \frac{2}{i(z + 2 - \sqrt{3})(z + 2 + \sqrt{3})} \]
        
        Soit $\gamma$ le cercle unité centré en $z = 0$ orienté positivement.
        
        \[ -1 < z_1 = \sqrt{3} - 2 < 0 \implies  z_1 \in \inte \gamma \]
        
        \[ z_2 = -(\sqrt{3} + 2) < -1 \implies  z_2 \notin \inte \gamma \]
        
        Donc
        
        \[ \int_0^{2\pi} \frac{\dth}{2 + \cos \theta} = 2\pi i \Res_{z_1} (\tilde{f}) \]
        
        $z_1 = \sqrt{3} - 2$ est un pôle d'ordre 1 de $\tilde{f}$, donc:
        
        \begin{align*}
        \Res_{z_1} (\tilde{f}) &= \lim_{z \rightarrow z_1} (z - z_1) \tilde{f}(z)
        \\&= \lim_{z \rightarrow \sqrt{3} - 2} (z + 2 - \sqrt{3}) \frac{2}{i(z + 2 - \sqrt{3})(z + 2 + \sqrt{3})}
        \\&= \frac{1}{i\sqrt{3}}
        \end{align*}
        
        Le résultat est:
        
        \[ \int_0^{2\pi} \frac{\dth}{2 + \cos \theta} = 2\pi i \frac{1}{i\sqrt{3}} = \frac{2\pi}{\sqrt{3}} \]
        
        \textit{Autres exemples: ex 1 à 4, série 10 et ex 1, série 11}
    \end{example}
\end{enumerate}

\subsection{Calcul d'intégrales généralisées}

\begin{enumerate}[label=\alph*)]
    \item But: calculer des intégrales de la forme
    
    \[ \int_{-\infty}^\infty f(x) e^{i\alpha x} \dx \quad \alpha \in \R_+,\ f: \R \rightarrow \R \]
\end{enumerate}
