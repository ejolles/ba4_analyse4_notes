% !TeX spellcheck = fr_FR
\chapter{Séries de Laurent, pôles et résidus}


\section{Polynôme et série de Taylor d'une fonction holomorphe}

\subsection{Définitions et résultats}

\begin{hypothesis}
    Soit un ouvert $D \subset \C$ et $f: \begin{array}{ccc}
    D & \longrightarrow & \C\\
    z & \longmapsto & f(z)
    \end{array}$ une fonction holomorphe dans $D$ et $z_0 \in D$.
\end{hypothesis}

\begin{definition}
    Pour $N \in \N$, le \textbf{polynôme de Taylor} de $f$ de degré $N$ en $z_0$ est:
    
    \[
    T_N f(z) = \sum_{n = 0}^N \frac{f^{(n)}(z_0)}{n!}(z - z_0)^n
    \]
\end{definition}

\begin{result}[séries de Taylor]
    Soit $R > 0$ et $D_R(z_0) = \{z \in \C : \module{z - z_0} < R \}$ le plus grand disque de rayon $R$ centré en $z_0$ contenu dans $D$.
    
    Convention: si $D = \C \implies R = +\infty$ et $D_R(z_0) = \C$
    
    Alors:
    
    \begin{enumerate}[label=\arabic{enumi})]
        \item 
        \[
        T f(z) = \lim_{N \rightarrow +\infty} T_N f(z) = \sum_{n = 0}^{+\infty} \frac{f^{(n)}(z_0)}{n!}(z - z_0)^n
        \]
        
        existe et est finie $\forall z \in D_R(z_0)$. L'expression $T f(z)$ s'appelle \textbf{la série de Taylor} de $f$ en $z_0$.
        
        \item 
        De plus, on a $f(z) = T f(z) \quad \forall z \in D_R(z_0)$
        
        $R$ est appelé \textbf{le rayon de convergence} de la série de Taylor.
        
        \item 
        Les coefficients de la série de Taylor sont reliés à la formule de Cauchy par le corollaire du §3.4.
        On a:
        
        \[ \frac{f^{(n)}(z_0)}{n!} = \frac{1}{2\pi i} \int_\gamma \frac{f(\xi)}{(\xi - z)^{n+1}} \dxi \]
        
        où $\gamma \subset D_R(z_0)$ est une courbe simple fermée régulière orientée positivement telle que $z_0 \in \inte \gamma$.
    \end{enumerate}
\end{result}

\subsection{Exemples}

\begin{example}[1]
    \[f(z) = e^z\]
    
    est holomorphe dans $\C$.
    On a $f^{(n)}(z) = e^z$ et $f^{(n)}(0) = 1 \quad \forall n \in N$.
    
    Donc:
    
    \[ e^z = \sum_{n=0}^{+\infty} \frac{z^n}{n!} \quad \forall z \in \C \]
\end{example}

\begin{example}[2]
    \[f(z) = \frac{1}{1-z}\]
    
     est holomorphe dans $D = \C \setminus \{1\}$.
    
    Le plus grand disque centré en $z_0 = 0$ contenu dans $D$ est $D_1(0) = \{ z \in \C : \module{z} < 1 \}$.
    
    On a $f^{(n)}(z) = \frac{n!}{(1-z)^{n+1}}$ et $f^{(n)}(0) = n! \ \forall n \in \N$.
    
    Donc:
    
    \[
    \frac{1}{1-z} = \sum_{n=0}^{+\infty} z^n \quad \forall z \in \C, \module{z} < 1
    \]
    
    \textbf{\og Série géométrique \fg{}} avec rayon de convergence $R = 1$.
\end{example}

\begin{example}[3]
    \[f(z) = \frac{1}{1 + z^2}\]
    
    est holomorphe dans $D = \C \setminus \{-i; i\}$.
    
    Le plus grand disque centré en $z_0 = 0$ et contenu dans $D$ est $D_1(0) = \{z \in \C : \module{z} < 1\}$.
    
    On a:
    
    \[
    \frac{1}{1 + z^2}
    = \frac{1}{1 - \left(-z^2\right)}
    \overset{\textrm{Ex. 2}}{=}
    \sum_{n=0}^{+\infty} (-z^2)^n
    = \sum_{n=0}^{+\infty} (-1)^n z^{2n} \quad \forall z \in \C, \module{z} < 1
    \]
    
    Le rayon de convergence $R = 1$
\end{example}

\textit{Autre exemple: ex. 5, série 7}






















