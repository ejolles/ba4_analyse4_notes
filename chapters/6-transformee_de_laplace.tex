% !TeX spellcheck = fr_FR
\chapter{Transformée de Laplace}


\section{Introduction}

\begin{motivation}
    Généralisation de la théorie de Fourier pour appliquer une étude de problèmes transitoires en électricité avec des conditions initiales.
\end{motivation}

\subsubsection*{Procédé heuristique induisant la définition de la transformée de Laplace:}

Soit $g: \R \rightarrow \R$, la transformée de Fourier $\TF (g): \R \rightarrow \Cx$ est définie par $\TF(g)(\alpha) = \ostpi \int_{-\infty}^{\infty} g(t) e^{-i\alpha t} \dt$.

On considère $g(t) = \left\{
\begin{array}{lcl}
f(t) & \textrm{ si } & t \geq 0\\
0    & \textrm{ si } & t < 0\\
\end{array}
\right.$ et on autorise la variable $\alpha \in \R$ à prendre des valeurs complexes.
En posant $\alpha = -i z$ avec $z \in \Cx$, on obtient:

\[ \sqrt{2\pi} \ \TF(g)(-iz) = \int_{-\infty}^{\infty} g(t) e^{-i(-iz)t} \dt = \int_{0}^{\infty} f(t) e^{-zt} \dt \]

Ceci est une nouvelle fonction qui dépend de la variable $z \in \Cx$ qui est appelée la transformée de Laplace de $f$.

\section{Transformée de Laplace d'une fonction}

\subsection{Définition}

\begin{definition}
    Soit $f: \R_+ \rightarrow \R$ une fonction continue par morceaux et soit $\gamma_0 \in \R$ tels que $\int_0^\infty \module{f(t)} e^{-\gamma_0 t} \dt < \infty$.
    
    La \textbf{transformée de Laplace} de $f$ est la fonction notée $\TL(f)$ où $F: \Cx \rightarrow \Cx$ définie par:
    
    \[ z \longmapsto \TL(f)(z) = F(z) = \int_0^\infty f(t) e^{-zt} \dt \quad \forall z \in \Cx : \Rep z \geq \gamma_0 \]
    
    $\gamma_0$ s'appelle l'abscisse de convergence de $f$.
\end{definition}

\begin{remark}
    Si $\Rep z \geq \gamma_0$, alors $\TL(f)(z)$ est bien définie.
    En effet, comme:
    
    \[ \module{e^{-zt}} = \module{e^{-(\Rep z + i \Imp z) t}} = e^{-t \Rep z} \module{e^{-i t \Imp z}} = e^{-t\Rep z} \leq e^{-t\gamma_0} \]
    
    pour $t \geq 0$ si $\Rep z \geq \gamma_0$.
    Alors:
    
    \[ \module{\TL(f)(z)} = \module{\int_0^\infty f(t) e^{-zt} \dt} \leq \int_0^\infty \module{f(t)} \module{e^{-zt}} \dt \leq \int_0^\infty \module{f(t)} e^{-\gamma_0 t} \dt < \infty \]
\end{remark}

\subsection{Exemples}

\begin{example}[1]
    Calculer la transformée de Laplace de la fonction $f: \R_+ \rightarrow \R$, où $f(t) = 1$.

Puisque

\begin{align*}
    \int_0^\infty \module{f(t)} e^{-\gamma_0 t} \dt
    &= \int_0^\infty e^{-\gamma_0 t} \dt
    = \left.-\frac{e^{-\gamma_0 t}}{\gamma_0}\right|_0^\infty
    \\&= \frac{1}{\gamma_0} \left[1 - \lim_{t \rightarrow \infty} e^{-\gamma_0 t}\right]
    = \left\{
    \begin{array}{clc}
    \frac{1}{\gamma_0} & \textrm{ si } \gamma_0 > 0\\
    +\infty & \textrm{ si } \gamma_0 < 0
    \end{array}\right.
\end{align*}

Alors n'importe quel $\gamma_0 \in \R_+^\ast$ est abscisse de convergence de $f$.

\[
    \TL(f)(z)
    = \int_0^\infty \module{f(t)} \module{e^{-zt}} \dt
    = \int_0^\infty \module{e^{-zt}} \dt
    = \left.-\frac{e^{-z t}}{z}\right|_0^\infty
    = \frac{1}{z} \left[1 - \lim_{t \rightarrow \infty} e^{-z t}\right]
\]

Or,

\[
    \lim_{t \rightarrow \infty} \module{e^{-zt}}
    = \lim_{t \rightarrow \infty} \module{e^{-(x + iy)t}}
    = \lim_{t \rightarrow \infty} e^{-xt} \module{e^{-iyt}}
    = 0
\]

si $x = \Rep z > 0$.

Résultat: $\TL(f): \Cx \rightarrow \Cx$, où:

\[
    z \longmapsto \TL(f)(z) = \frac{1}{z} \textrm{ si } \Rep z > 0
\]
\end{example}

\begin{example}[2]
Calculer la transformée de Laplace de la fonction $f: \R_+ \rightarrow \R$, où $f(t) = e^{at}$ avec $a \in \R$:

Comme 

\begin{align*}
    \int_0^\infty \module{f(t)} e^{-\gamma_0 t} \dt 
    &= \int_0^\infty e^{(a - \gamma_0)t}
    = \left.
    \frac{e^{(a - \gamma_0)t}}{a - \gamma_0}
    \right|_0^\infty
    \\& = \frac{1}{a - \gamma_0} \left[ \lim_{t \rightarrow \infty} e^{(a - \gamma_0)t} - 1 \right]
    = \left\{
    \begin{array}{clc}
    +\infty & \textrm{ si } a \geq \gamma_0\\
    \frac{1}{\gamma_0 - a} & \textrm{ si } a < \gamma_0
    \end{array}\right.
\end{align*}

Alors, n'importe quel $\gamma_0 > a$ est abscisse de convergence de $f$.

\[
    \TL(f)(z)
    = \int_0^\infty f(t) e^{-z t} \dt 
    = \int_0^\infty e^{(a - z)t}
    = \left.
    \frac{e^{(a - z)t}}{a - z}
    \right|_0^\infty
    = \frac{1}{a - z} \left[ \lim_{t \rightarrow \infty} e^{(a - z)t} - 1 \right]
\]

Or,

\[
    \lim_{t \rightarrow \infty} \module{e^{(a - z)t}}
    = \lim_{t \rightarrow \infty} \module{e^{(a - x - iy)t}}
    = \lim_{t \rightarrow \infty} e^{(a - x)t} \module{e^{- iyt}}
    = 0
\]

si $a - x < 0$.

Résultat: $\TL(f): \Cx \rightarrow \Cx$, où:

\[
z \longmapsto \TL(f)(z) = \frac{1}{z - a} \textrm{ si } \Rep z > a
\]

\textit{Autres exemples: ex.1, série 12}

\end{example}


\section{Propriétés de la transformée de Laplace}

On considère deux fonctions, $f,g: \R_+ \rightarrow \R$ continues par morceaux et $\gamma_0 \in \R$ tels que $\int_{0}^{\infty} \module{f(t)} e^{-\gamma_0 t} \dt < \infty$ et $\int_{0}^{\infty} \module{g(t)} e^{-\gamma_0 t} \dt < \infty$.

On note $\TL(f) = F$ et $\TL(g) = G$ les transformées de Laplace de $f$ et de $g$.

\subsection{Linéarité et décalage}

\begin{itemize}
    \item 
    $\TL(af + bg) = a \TL(f) + b \TL(g)$ avec $a, b \in \R$
    \item 
    Si $a \in \R_+^\ast$ et $b \in \R$ et $h(t) = e^{-bt} f(at)$ alors
    
    \[ \TL(h)(z) = \frac{1}{a} \TL(f)(\frac{z + b}{a}) \]
    
    pour tout $z \in \Cx$ tel que $\Rep\left(\frac{z + b}{a}\right) \geq \gamma_0$
\end{itemize}

\subsection{Transformée de Laplace du produit de convolution}

Si

    \[
        (f \ast g)(t)
        = \int_{-\infty}^\infty f(t - s)g(s) \ds
        = \int_0^t f(t-s)g(s) \ds
    \]

est le produit de convolution de $f$ et de $g$, alors:

\[
    \TL(f \ast g)(z) = \TL(f)(z)\TL(g)(z) \quad \forall z \in \Cx : \Rep z \geq \gamma_0
\]

\subsection{Holomorphie et dérivation de la transformée de Laplace}

$\TL(f)$ est holomorphe dans le domaine $D = \{ z \in \Cx | \Rep z > \gamma_0 \}$.
De plus, $\forall z \in D$, on a:

\[ \TL(f)'(z) = - \int_0^\infty t f(t) e^{-zt} \dt = - \TL(h)(z) \]

où $h: \R_+ \rightarrow \R$ est définie par $h(t) = tf(t)$.

\subsection{Transformée de Laplace de la dérivée d'une fonction}

Si de plus $f \in C^1(\R_+)$ et $\int_0^\infty \module{f'(t)} e^{-\gamma_0 t} \dt < \infty$, alors $\forall z \in D$, on a:

\[
    \TL(f')(z) = z \TL(f)(z) - f(0)
\]

Plus généralement: si $f \in C^n (R_+)$ et $\int_0^\infty \module{f^{(k)}(t)} e^{-\gamma_0 t} \dt < \infty$ pour $k = 1, 2, \ldots, n$, alors:

\[
    \TL(f^{(n)})(z) = z^n \TL(f)(z) - z^{n-1} f(0) - z^{n-2} f'(0)
    - \ldots - z f^{(n-2)}(0) - f^{(n-1)}(0) \quad \forall z \in D
\]

\subsection{Transformée de Laplace d'une primitive d'une fonction}

Si de plus $f \in C(\R_+)$ avec $\gamma_0 \geq 0$ et si $\varphi(t) = \int_0^t f(s) \ds$ est une primitive de $f$, alors $\forall z \in D$, on a:

\[
    \TL(\varphi)(z) = \frac{1}{z} \TL(f)(z)
\]

\subsection{Esquisse des démonstrations des propriétés}

\textit{Cf. ex.3, série 12}

\subsection{Exemples d'utilisation des propriétés}

\begin{example}
    Calculer la transformée de Laplace de la fonction $f: \R_+ \rightarrow \R$ définie par $f(t) = t^2$.
    
    Méthode: utiliser la propriété 6.3.4 de la deuxième dérivée de $f$:
    
    On a $\TL(f'')(z) = z^2 \TL(f)(z) - zf(0) - f'(0)$.
    Comme $f(t) = t^2, \enspace f'(t) = 2t$ et $f''(t) = 2 \implies f(0) = 0$ et $f'(0) = 0$, et $f''(t) = 2g(t)$ avec $g(t) = 1$ si $t \geq 0$.
    
    \[ \implies \TL(f'')(z) = 2 \TL(g)(z) = 2 \frac{1}{z} \]
    
    Donc on obtient $\frac{2}{z} = z^2 \TL(f)(z) \implies \TL(f)(z) = \frac{2}{z^3}$
    
    Résultat, pour $\TL(f): \Cx \rightarrow \Cx$:
    
    \[ z \longmapsto \TL(f)(z) = \frac{2}{z^3} \]
    
    \textit{Autres exemples: ex. 2, série 12}
\end{example}


\section{La formule d'inversion de la transformée de Laplace}

\subsection{Théorème de la transformée de Laplace inverse}

\begin{theorem}
    Soit $f: \R_+ \rightarrow \R$ une fonction (étendue à $\R$ en posant $f(t) = 0$ pour $t < 0$) continue pour $t > 0$ et soit $\gamma_0 \in \R$ tel que $\int_0^\infty \module{f(t)}e^{-\gamma_0 t}\dt < \infty$.
    
    Si la transformée de Laplace $F = \TL(f)$ de $f$ est telle que $\int_{-\infty}^\infty F(\gamma + is) \ds < \infty$ pour un certain $\gamma > \gamma_0$, alors on a la formule d'inversion de $\TL$:
    
    \[ \TLi(F)(t) := \frac{1}{2\pi} \int_{-\infty}^\infty F(\gamma + is) e^{(\gamma + is) t} \ds = f(t) \quad \forall t > 0 \]
    
    $\TLi$ s'appelle la transformée de Laplace inverse.
    L'intégrale de la définition est indépendante de $\gamma$.
\end{theorem}

\subsection{Exemples d'utilisation}

\begin{example}[1]
    Trouver une fonction $f: \R_+ \rightarrow \R$ telle que sa transformée de Laplace soit:
    
    \[ F(z) = \frac{1}{(z-a)(z-b)} \]
    
    où $a, b \in \R, a \neq b$.
    Autrement dit: trouver $\TLi(F)$.
    
    On décompose $F$ en élément simples.
    
    \begin{align*}
        F(z) &= \frac{\alpha}{z - a} + \frac{\beta}{z - b}
        = \frac{\alpha (z - b) + \beta (z-a)}{(z-a)(z-b)}
        = \frac{(\alpha + \beta) z - (\alpha b + \beta a)}{(z-a)(z-b)}
        \\&\implies \beta = \frac{1}{b - a}, \alpha = \frac{1}{a - b}
    \end{align*}
    
    On a donc:
    
    \[ F(z) = \frac{1}{a - b} \left[\frac{1}{z - a} - \frac{1}{z - b}\right] = \frac{1}{a - b} \left[G(z) - H(z)\right] \]
    
    avec $G(z) = \frac{1}{z - a}$ et $H(z) = \frac{1}{z - b}$.
    Alors:
    
    \begin{align*}
        f(t) &= \TLi(F)(t)
        = \TLi\left(\frac{1}{a - b} \left[G(z) - H(z)\right]\right)(t)
        \\&= \frac{1}{a - b} \left[ \TLi(G)(t) - \TLi(H)(t) \right]
        = \frac{1}{a - b} \left[ e^{at} - e^{bt} \right]
    \end{align*}
    
    \textit{Voir exemple 2 §6.2.2}
\end{example}


\begin{example}[2]
    Trouver une fonction $f: \R_+ \rightarrow \R$ telle que sa transformée de Laplace soit:
    
    \[ F(z) = \frac{z^2}{(z^2+1)^2} \]
    
    Autrement dit: trouver $\TLi(F)$.
    
    On remarque:
    
    \begin{align*}
    F(z) &= \frac{z^2}{(z^2+1)^2}
    = \frac{z^2 - 1 + z^2 + 1}{2(z^2+1)^2}
    = \frac{z^2 - 1}{2(z^2+1)^2} + \frac{z^2 + 1}{2(z^2+1)^2}
    \\&= \frac{z^2 - 1}{2(z^2+1)^2} + \frac{1}{2(z^2+1)}
    = \frac{1}{2} G(z) + \frac{1}{2} H(z)
    \end{align*}
    
    avec $G(z) = \frac{z^2 - 1}{(z^2+1)^2}$ et $H(z) = \frac{1}{z^2+1}$.
    
    Alors:
    
    \begin{align*}
    f(t) &= \TLi(F)(t)
    = \TLi\left(\frac{1}{2} \left[G(z) + H(z)\right]\right)(t)
    \\&= \frac{1}{2} \left[ \TLi(G)(t) + \TLi(H)(t) \right]
    = \frac{1}{2} \left[ t \cos t +  \sin t \right]
    \end{align*}
\end{example}


\begin{example}[3]
    Trouver une fonction $f: \R_+ \rightarrow \R$ telle que sa transformée de Laplace soit:
    
    \[ F(z) = \frac{z^2}{(z+1)(z-2)^2} \]
    
    Autrement dit: trouver $\TLi(F)$.
    
    \textbf{Méthode:} utiliser le théorème des résidus pour la fonction $h$ définie par:
    
    \[ h(z) = F(z) e^{zt} \]
    
    avec $t > 0$ fixé.
    
    \begin{enumerate}
        \item 
        On cherche les singularités de $h(z) = \frac{z^2 e^{zt}}{(z+1)(z-2)^2}$.
        $z = -1$ est un pôle d'ordre 1 et $z = 2$ est un pôle d'ordre 2 de $h$.
        
        \item 
        On choisit $\gamma \in \R$ tel que toutes les singularités de $h$ se trouvent à gauche de la droite $\Rep z = \gamma$.
        
        \item 
        On choisit $r > 0$ assez grand pour que le cercle $C_r$ de rayon $r$ et centré à l'origine intercepte la droite $\Rep z = \gamma$ et que toutes les singularités de $h$ soient à l'\textit{intérieur} de la courbe fermée $\gamma_r = C'_r \cup L_r$ où
        
        \[ C'_r = \left\{ z \in \Cx : |z| = r \ \textrm{ et } \Rep z < \gamma \right\} \]
        
        \[ L_r = \left\{ z \in \Cx : \Rep z = \gamma \ \textrm{ et } \module{\Imp z} \leq \sqrt{r^2 - \gamma^2} \right\} \]
        
        \item 
        On applique le théorème des résidus à $h$ et $\gamma_r$.
        
        \[ \int_{\gamma_r} \frac{z^2 e^{zt}}{(z+1)(z-2)^2} \dz = 2 \pi i \left[ \Res_{-1}(h) + \Res_2(h) \right] \]
        
        On a:
        
        \[
            \Res_{-1}(h)
            = \lim_{z \rightarrow -1} (z + 1) h(z)
            = \lim_{z \rightarrow -1} \frac{z^2 e^{zt}}{(z-2)^2}
            = \frac{e^{-t}}{9}
        \]
        
        \begin{align*}
            \Res_2(h)
            &= \lim_{z \rightarrow 2} \frac{\dd}{\dz} \left[ (z - 2)^2 h(z) \right]
            = \lim_{z \rightarrow 2} \frac{\dd}{\dz} \frac{z^2 e^{zt}}{z+1}
            \\&= \frac{1}{9} \left( 8 e^{2t} + 12t e^{2t} \right)
            = (8 + 12t) \frac{e^{2t}}{9}
        \end{align*}
        
        Donc:
        
        \[ \int_{\gamma_r} \frac{z^2 e^{zt}}{(z+1)(z-2)^2} \dz = \frac{2 \pi i}{9} \left[ e^{-t} + (8 + 12t) e^{2t} \right] \]
        
        \item 
        On a aussi:
        
        \[
            \int_{\gamma_r} \frac{z^2 e^{zt}}{(z+1)(z-2)^2} \dz
            = \int_{C'_r} F(z) e^{zt} \dz + \int_{L_r} F(z) e^{zt} \dz
        \]
        
        Par ailleurs:
        \begin{itemize}
            \item 
            Si $\module{F(z)} \leq \frac{C}{|z|^k}$ avec $C \in \R$ et $k > 0$ pour $z \in C'_r$ avec $|z|$ suffisamment grand, alors on peut montrer que $\lim\limits_{r \rightarrow \infty} \int_{C'_r} F(z)e^{zt} \dz = 0$.
            
            Pour $F(z) = \frac{z^2}{(z+1)(z-2)^2}$, la condition est vérifiée avec $k = 1$.
            
            \item 
            \begin{align*}
                \lim_{r \rightarrow \infty} \int_{L_r} F(z) e^{zt} \dz
                &= \lim_{r \rightarrow \infty}
                i \int_{-\sqrt{r^2 - \gamma^2}}^{\sqrt{r^2 - \gamma^2}} F(\gamma + is) e^{(\gamma + is)t} \ds
                \\&= i \int_{-\infty}^\infty F(\gamma + is) e^{(\gamma + is)t} \ds
                = 2 \pi i \TLi(F)(t)
            \end{align*}
            
            avec $z = \gamma + is$, $\dz = i \ds$.
            On obtient la formule d'inversion de la transformée de Laplace.
            
            \item 
            \[
                \lim\limits_{r \rightarrow \infty} \int_{\gamma_r} \frac{z^2 e^{zt}}{(z+1)(z-2)^2} \dz
                = \frac{2 \pi i}{9} \left[ e^{-t} + (8 + 12t) e^{2t} \right]
            \]
        \end{itemize}
    \end{enumerate}

Finalement, lorsque $r \rightarrow \infty$, on obtient:

\[
    2 \pi i \TLi(F)(t) = \frac{2 \pi i}{9} \left[ e^{-t} + (8 + 12t) e^{2t} \right]
\]

D'après le théorème de la transformée de Laplace inverse, on a:

\[
    f(t) = \TLi(F)(t)
    = \frac{1}{9} \left[ e^{-t} + (8 + 12t) e^{2t} \right]
\]
\end{example}

\textit{Autres exemples: ex 1-2, série 13}


\section{Quelques applications de la transformée de Laplace}

\begin{enumerate}[label=\alph{enumi})]
    \item 
    Trouver une solution d'équations intégrales du type produit de convolution \textit{(cf ex.3, série 13)}.
    
    \item 
    Trouver une solution d'équations différentielles du type
    
    \[ a y''(t) + b y'(t) + c y(t) = f(t) \]
    
    pour $t > 0$ avec les conditions initiales $y(0)$ et $y'(0)$ données.
    
    \begin{example}
        Trouver une solution $y$ de:
        
        \[ y''(t) + 4 y'(t) + 3 y(t) = 0 \]
        
        pour $t > 0$ avec conditions initiales $y(0) = 3$ et $y'(0) = 1$.
        
        On écrit:
        
        \[ \TL(y'' + 4 y' + 3 y)(t) = \TL(f)(z) \]
        
        avec $f(t) = 0$ pour $t > 0$.
        
        \[ \implies \TL(y'')(z) + 4 \TL(y')(z) + 3 \TL(y)(z) = 0 \]
        
        car $\TL(f)(z) = 0$.
        En écrivant $\TL(y)(z) = Y(z)$, on a:
        
        \[ \implies \left[ z^2 Y(z) - z y(0) - y'(0) \right] + 4 \left[ z Y(z) - y(0) \right] + 3 Y(z) = 0 \]
        
        \[ \implies \left[ z^2 Y(z) - 3z - 1 \right] + 4 \left[ z Y(z) - 3 \right] + 3 Y(z) = 0 \]
        
        \[ \implies \left[ z^2 + 4z + 3 \right] Y(z) - 3z - 13 = 0 \]
        
        \[ \implies Y(z) = \frac{3z + 13}{z^2 + 4z + 3} = \frac{3z + 13}{(z+3)(z+1)} \]
        
        On calcule la transformée de Laplace inverse de $Y(z)$.
        
        Méthode des résidus (conditions vérifiées avec $k = 1$) avec:
        
        \[ h(z) = Y(z) e^{zt} = \frac{(3z + 13)e^{zt}}{(z+3)(z+1)} \]
        
        Deux pôles d'ordre 1: $z = -3$ et $z = -1$
        
        \[ \Res_{-3}(h) = \lim_{z \rightarrow -3} (z + 3)h(z)
        = \lim_{z \rightarrow -3} \frac{(3z + 13)e^{zt}}{z+1}
        = \frac{4 e^{-3t}}{-2} = -2 e^{-3t} \]
        
        \[ \Res_{-1}(h) = \lim_{z \rightarrow -1} (z + 1)h(z)
        = \lim_{z \rightarrow -1} \frac{(3z + 13)e^{zt}}{z+3}
        = \frac{10 e^{-t}}{2} = 5 e^{-t} \]
        
        La solution est
        
        \[ y(t) = \TLi(Y)(t) = \Res_{-3}(h) + \Res_{-1}(h) = 5 e^{-t} - 2 e^{-3t} \]
        
        \textit{Autre exemple: ex. 4, série 13}
    \end{example}
\end{enumerate}
